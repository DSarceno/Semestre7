\section{Problema 1}
Dado que la base $\{ \ket{\alpha _i} \}$ son kets propios de los observables $A$ y $B$. Entonces aplicamos el conmutador a un vector cualquiera de la base
	$$ [A,B]\ket{\alpha _i} = AB\ket{\alpha _i} - BA\ket{\alpha _i}, $$
	$$ [A,B]\ket{\alpha _i} = A(b_i \ket{\alpha _i}) - B(a_i \ket{\alpha _i}), $$
	$$ [A,B]\ket{\alpha _i} = a_i b_i \ket{\alpha _i} - b_i a_i \ket{\alpha _i} = 0. $$
	Lo que demuestra que los observables $A$ y $B$ son compatibles.

\section{Problema 2}
Dada la transformada de fourier
	$$ \overset{\sim}{\psi} (p) = \frac{1}{\sqrt{2\pi \hbar}} \int _{-\infty} ^\infty e^{-\frac{ixp}{\hbar}} \psi (x) \dd{x}. $$
Derivando esa expresión respecto a $p$
	\begin{equation}
		\pdv{\overset{\sim}{\psi} (p)}{p} = \frac{1}{i\hbar} \frac{1}{\sqrt{2\pi \hbar}} \int _{-\infty} ^\infty e^{-\frac{ixp}{\hbar}} x\psi (x) \dd{x}. \label{psicolochop}
	\end{equation}
Ahora, tomando
	\begin{equation}
		\mel{p}{X}{\psi} = \int _{-\infty} ^\infty \braket{p}{x} \mel{x}{X}{\psi} = \frac{1}{\sqrt{2\pi \hbar}} \int _{-\infty} ^\infty e^{-\frac{ixp}{\hbar}} x\psi (x) \dd{x}. \label{sandwich}
	\end{equation}
Sustituyendo \eqref{sandwich} en \eqref{psicolochop}, se tiene
	$$\boxed{i\hbar \pdv{\overset{\sim}{\psi} (p)}{p} = \mel{p}{X}{\psi}.}$$

\section{Problema 3}
Tomando
	$$ \mel{p}{e^{-\frac{ia}{\hbar} X}}{\psi} = \int _{-\infty} ^\infty \braket{p}{x} \mel{p}{e^{-\frac{ia}{\hbar} X}}{\psi} = \underbrace{\int _{-\infty} ^\infty \frac{1}{\sqrt{2\pi \hbar}} e^{-\frac{ixp}{\hbar}} \psi (x + a) \dd{x}}_{\text{Transformada de Fourier } = \, \overset{\sim}{\psi} (p + a)}. $$
	$$ \boxed{\mel{p}{e^{-\frac{ia}{\hbar} X}}{\psi} = \overset{\sim}{\psi} (p + a)} $$


\section{Problema 4}
Tomando el conmutador y la serie de taylor de la exponencial
	$$e^{P^2} = \sum _n ^\infty \frac{P^{2n}}{n!},$$
entonces, reemplazando en el conmutador dado
	$$ [X,e^{P^2}] = \sum _n ^\infty \frac{1}{n!} [X,P^{2n}] = \sum _n ^\infty \frac{2n}{n!} (i\hbar) P^{2n - 1} = 2i\hbar P \sum _n ^\infty \frac{P^{2(n - 1)}}{(n - 1)!}, $$
por lo que 
	$$\boxed{ [X,e^{P^2}] = 2i\hbar P e^{P^2} }$$

\section{Problema 5}
Para el conmutador $[P,e^{X^2}]$, tomamos la misma idea del ejercicio pasado junto con la propiedad $[A,B] = -[B,A]$, entonces
	$$ [P,e^{X^2}] = -\sum _n ^\infty \frac{1}{n!} [X^{2n} ,P] = -2i\hbar X \sum _n ^\infty \frac{X^{2(n - 1)}}{(n - 1)!}, $$
de modo que
	$$ \boxed{[P,e^{X^2}] = -2i\hbar Xe^{X^2}} $$


\section{Problema 6}
Dados dos observables $A$ y $B$ compatibles. Dado $\ket{\phi}$ es vector propio de $A$, se tiene la siguiente igualdad
	$$ \underbrace{[A,B]}_{0} \ket{\phi} = AB \ket{\phi} - BA\ket{\phi} = \underbrace{(A - aI)}_{\neq 0} \underbrace{B\ket{\phi}}_{\neq 0}, $$
entonces, $B\ket{\phi}$ debe ser una constante por el ket dado, de modo que $A(b\ket{\phi}) - ab\ket{\phi} = (ab - ab)\ket{\phi} = 0$, entonces $\ket{\phi}$ es vector propio de $B$.


















%%%%%%%5