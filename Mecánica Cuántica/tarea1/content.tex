\section{Problema 1}
Tomando dos estados $\ket{\psi} ,\ket{\phi} \in \Hilbert$ y dos operadores hermíticos $A,B$, entonces, por definición, se tiene
	\begin{align*}
		\mel{\psi}{A}{\phi} &= \mel{\phi}{A}{\psi} ^* \\
		\mel{\psi}{B}{\phi} &= \mel{\phi}{B}{\psi} ^*		
	\end{align*}
	por linealidad, sumamos ambas expresiones
	\begin{align*}
		\bra{\psi} \qty(A\ket{\phi} + B\ket{\phi}) &= \qty[\bra{\phi} \qty(A\ket{\psi} + B\ket{\psi})]^* \\
		\mel{\psi}{(A+B)}{\phi} &= \mel{\phi}{(A+B)}{\psi} ^* .
	\end{align*}
	Con esto, queda demostrado que $(A + B) = (A + B)^\dagger$
	
\section{Problema 2}
Teniendo dos operadores unitarios $U_1$ y $U_2$, tomamos el nuevo operador $U_1 U_2$, con esto, aplicamos su operador adjunto
	\begin{align*}
		U_1 U_2 (U_1 U_2)^\dagger &= U_1 \underbrace{U_2 U_2 ^\dagger}_{=I} U_1 ^\dagger \\
		U_1 U_1 ^\dagger &= I,
	\end{align*}
	esto se cumple por propiedades de los operadores unitarios, y se demuestra que $U_1 U_2$ es unitario.
	
\section{Problema 3}
Dado $C = i[A,B]$ con $A,B$ hermíticos, tomando el adjunto de $C$
	\begin{align*}
		C^\dagger = i^* [A,B]^\dagger &= -i\qty((AB)^\dagger - (BA)^\dagger) \\
		&= -i\qty(B^\dagger A^\dagger - A^\dagger B^\dagger) \\
		&= -i\qty(BA-AB) \\
		&= i\qty(AB-BA) = C.
	\end{align*}
	Con esto queda demostrado que le operador $C$ es hermítico.
	
\section{Problema 4}
Dados $A$ y $B$ hermíticos:
	\begin{enumerate}[a)]
		\item $(\Rightarrow)$ dado que $AB = (AB)^\dagger$, tomando el lado derecho $(AB)^\dagger = B^\dagger A^\dagger = BA$, esto implica que $[A,B] = 0$, por lo que conmutan. $\hfill \square$ \\
		$(\Leftarrow)$ Dado que $[A,B] = 0$, entonces $0 = AB - BA = AB - B^\dagger A^\dagger = AB - (AB)^\dagger = 0$, lo que implica que $AB$ es hermítico. $\hfill \square$
		\item Dado $(A+B)^n$ tomamos el binomio de newton y aplicamos el adjunto
		\begin{align*}
			(A + B)^{n ^\dagger} &= \sum _{k = 0} ^n \mqty(n \\ k) \qty(A^{n-k} B^k)^\dagger \\
			&= \sum _{k = 0} ^n \mqty(n \\ k) (B^\dagger)^k (A^\dagger)^{n-k}
		\end{align*}
		renombrando los exponentes (esto se puede hacer puesto solo es cambiar el orden de las sumas y los operadores son conmutativos respecto a la suma)
		\begin{align*}
			&= \sum _{k = 0} ^n \mqty(n \\ k) B^{n - k} A ^k = (B + A)^n = (A + B)^n
		\end{align*}
		entonces, $(A + B)^n$ es hermítico $\hfill \square$
	\end{enumerate}
	
\section{Problema 5}
Dado un operador $A$
	\begin{align*}
		(A + A^\dagger) ^ \dagger = A^\dagger + \qty(A^\dagger)^\dagger = A^\dagger + A \quad \quad \quad \quad \square
	\end{align*}
	y
	\begin{align*}
		\qty[i(A - A^\dagger)]^\dagger = i^* (A^\dagger - A) = i(A - A^\dagger).
	\end{align*}
\section{Problema 6}
Dado un $A$ hermítico, entonces $\qty(e^{iA})^\dagger = e^{-iA^\dagger} = e^{-iA}$ esto es fácil ver debido a la expanción Taylor de la exponencial. Con esto
	\begin{align*}
		\qty(e^{iA}) \qty(e^{-iA}) = I.
	\end{align*}
	
\section{Problema 7}
Tomando un estado cualquiera $\ket{\phi} = a_j \ket{\alpha _j}$, entonces definimos $A = \sum _i \ketbra{\alpha _i}{\alpha _i}$, aplicando $A$ al estado, se tiene
	$$ A\ket{\phi} = \sum _i \sum _j a_j \ket{\alpha_i} \underbrace{\bra{\alpha _i} \ket{\alpha _j}}_{\delta _{ij}}. $$
	Dado que la delta de kronecker es $1$ solo cuando $i=j$, entonces simplificando la expresión se tiene
	$$ \Rightarrow \quad \sum _i a_i \ket{\alpha _i} = \ket{\phi}, $$
	lo que implica que $A = I$ $\hfill \square$
	
\section{Problema 8}
Tomando a ambos vectores en términos de una base ortonormal del espacio y los coeficientes del Fourier ($a_j = \braket{e_j}{\psi}$)
	\begin{align*}
		\ket{\phi} &= \sum _\alpha \braket{\mu _\alpha}{\phi} \ket{\mu _\alpha} \\
		\ket{\psi} &= \sum _\beta \braket{\mu _\beta}{\psi} \ket{\mu _\beta}
	\end{align*}
Con esto, encontramos el bra del estado $\ket{\phi}$, aplicando el nuevo funcional al vector $\ket{\psi}$, se tiene
	\begin{align*}
		\braket{\phi}{\psi} &= \sum _\alpha \sum _\beta \qty(\braket{\phi}{\mu _\alpha} \bra{\mu _\alpha}) \qty(\braket{\mu _\beta}{\psi} \ket{\mu _\beta}) \\
		&= \sum _\alpha \sum _\beta \braket{\phi}{\mu _\alpha} \braket{\mu _\beta}{\psi} \underbrace{\braket{\mu _\alpha}{\mu _\beta}}_{\delta_{\alpha \beta}} .
	\end{align*}
Dada la delta de kroneker, los términos de la sumatoria que permanecen son aquellos cuyos índices $\alpha = \beta$, entonces solo permanece una de las dos sumatorias
	\begin{align*}
		\braket{\phi}{\psi} &= \sum _\alpha \braket{\phi}{\mu _\alpha} \braket{\mu _\alpha}{\psi} \\
		\braket{\phi}{\psi} &= \bra{\phi}\underbrace{\qty(\sum _\alpha \ketbra{\mu _\alpha}{\mu _\alpha})}_{I} \ket{\psi} .
	\end{align*}
Lo que demuestra lo solicitado.

\section{Problema 9}

Dado que la el operador cumple con $H^4 = I$, aplicamos el operador $4$ veces a un vector propio cualquiera, con lo que se tiene
	$$H^4 \ket{\psi _n} = a_n ^4 \ket{\psi _n} = I\ket{\psi _n},$$
	con esto, se tiene la ecuación algebraica 
	\begin{equation}
		a_n ^4 - 1 = 0. \label{Eq9}
	\end{equation}
\begin{enumerate}[a)]
	\item Teniendo \eqref{Eq9} y que $H$ es hermítico (valores propios reales), las únicas soluciones admitidas son $\boxed{a_n = \pm 1}$.
	\item Para el caso en el que $H$ no es hermítico, se tiene que $\boxed{a_n = \pm 1, \quad a_n = \pm i}$.
\end{enumerate}
















%%%%%%%5