\input{Preamble_general.tex}




% para los metadatos del PDF
%\usepackage[%
%bookmarksnumbered,%
%pdfauthor={Diego Sarceño (dsarceno68@gmail.com)},%
%pdftitle={Puntos de Lagrange},%
%pdfsubject={Proyecto},%
%pdfkeywords={template, template}]{hyperref}

\title{\sc Resumen Artículos del arXiv qtm-ph}%
\author{Diego Sarceño \\ $201900109$}
\date{Guatemala, \today}
%% 20210307

\begin{document}  
\maketitle

%\begin{abstract}
%  \lipsum[1]
%\end{abstract}


%\section{Introducción}
%\label{sec:intro}
%\justify 
%\lipsum[1]	


\section{Erasure tolerant quantum memory and the quantum null energy condition in holographic systems (\textit{arXiv: 2202.00022v1})}
\justify
%\noindent
\lipsum[1]

\section{Algebraic Bethe Circuits (\textit{arXiv: 2202.04673v1})}
\justify
\lipsum[1]

\section{Self$-$Adjointness of Toeplitz Operators on the Segal$-$Bargmann Space (\textit{arXiv: 2202.04687v1})}
\justify
\lipsum[1]

\section{ (\textit{arXiv: })}
\justify
\lipsum[1]

\section{ (\textit{arXiv: })}
\justify
\lipsum[1]

\section{ (\textit{arXiv: })}
\justify
\lipsum[1]

\section{ (\textit{arXiv: })}
\justify
\lipsum[1]

\section{ (\textit{arXiv: })}
\justify
\lipsum[1]


\section{Anexos}
\label{sec:anexos}



%\section*{Agradecimientos}
%\label{sec:agradecimientos}


% References
\nocite{*}
\bibliographystyle{IEEEannot}%
\bibliography{references}%

\begin{thebibliography}{00}
\bibitem{b1} R. Symon, \textit{Mechanics} 3a. Ed. Addison$-$Wesley Publishing Company, 1971
\bibitem{b2} R. Taylor, \textit{Classical Mechanics}, Edwards Brothers, Inc. 2005.
\end{thebibliography}

\end{document}




%%% Local Variables:
%%% mode: latex
%%% TeX-master: t
%%% End:
