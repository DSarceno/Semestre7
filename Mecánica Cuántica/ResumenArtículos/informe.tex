\input{Preamble_general.tex}




% para los metadatos del PDF
%\usepackage[%
%bookmarksnumbered,%
%pdfauthor={Diego Sarceño (dsarceno68@gmail.com)},%
%pdftitle={Puntos de Lagrange},%
%pdfsubject={Proyecto},%
%pdfkeywords={template, template}]{hyperref}

\title{\sc Resumen de 8 Artículos de arXiv qtm-ph}%
\author{Diego Sarceño \\ $201900109$}
\date{Guatemala, \today}
%% 20210307

\begin{document}  
\maketitle

\begin{abstract}
  \lipsum[1]
\end{abstract}


\section{Introducción}
\label{sec:intro}
\justify 
\lipsum[1]	


\section{Puntos de Lagrange}
\justify
%\noindent
\lipsum[1-5]



\section{Anexos}
\label{sec:anexos}



%\section*{Agradecimientos}
%\label{sec:agradecimientos}


% References
\nocite{*}
\bibliographystyle{IEEEannot}%
\bibliography{references}%

\begin{thebibliography}{00}
\bibitem{b1} R. Symon, \textit{Mechanics} 3a. Ed. Addison$-$Wesley Publishing Company, 1971
\bibitem{b2} R. Taylor, \textit{Classical Mechanics}, Edwards Brothers, Inc. 2005.
\end{thebibliography}

\end{document}




%%% Local Variables:
%%% mode: latex
%%% TeX-master: t
%%% End:
