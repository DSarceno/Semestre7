\section{Notación de Dirac (Kets y Bras)}

Los estados en Mecánica Cuántica se describen matemáticamente por vectores. \\

Sea $\Hilbert$ un espacio vectorial sobre un campo de escalares $\F$, pero en Mecánica Cuántica el campo de escalares es el campo de los complejos ($\C$).

\dsnote{aqui falta xd}

Se define el Ket como
	$$\ket{\phi} \in \Hilbert$$
Si $\lambda$ es un escalar, osea que $\lambda \in \F$

	$$\lambda \phi = \ket{\lambda \phi} = \lambda \ket{\phi}$$
	
	
\subsection{Espacio Dual}
Sea $\Hilbert ^*$ el espacio dual algebraico de $\Hilbert$, entonces
	$$\Hilbert ^* = \{ \psi :\Hilbert \to \C | \psi (au + bv) = a\psi (u) + b\psi (v), \, \forall \, a,b \in \F \, \land \, u,v \in \Hilbert  \}$$

En la notación de Dirac a los elementos del espacio dual $\Hilbert ^*$ los vamos a llamar vectores bra. Los cuales son escritos de la siguiente forma:
	Si $\psi \in \Hilbert ^*$
		$$\bra{\psi}$$
	
	
	
	
	
	
	
	
	
	
	
	
	