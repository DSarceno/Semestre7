\section{Probabilidad}
\subsection{Probabilidad Clásica}

\begin{equation}
	P(A) = \frac{\abs{A}}{\abs{\Omega}} \label{prob_clasica}
\end{equation}

¿Qué problemas tiene la probabilidad  clásica?

La definición conocida de probabilidad clásica puede aplicarse cuando:
\begin{itemize}
	\item El espacio meustral es finito.
	\item Todos los elementos del espacio muestral tienen el mismo peso.
\end{itemize}

\paragraph{Propiedades de la Probabilidad Clásica}

\begin{itemize}
	\item $P(\Omega) = 1$.
	\item $P(A) \geq 0$ para cualquier evento $A$.
	\item $P(A\cup B) = P(A) + P(B)$ si $A$ y $B$ son disjuntos.
\end{itemize}

\subsection{Probabilidad Geométrica}

Si un experimento aleatorio tiene como espacio muestral $\Omega \subset \R ^2$ cuya área está bien definida y es finita, entonces se define la probabilidad geométrica de un evento $A \subseteq \Omega$ como
\begin{equation}
	P(A) = \frac{\text{Área de } A}{\text{Área de } \Omega} \label{prob_geometrica}	
\end{equation}


\paragraph{Propiedades de la Probabilidad Geométrica}

\begin{itemize}
	\item $P(\Omega) = 1$.
	\item $P(A) \geq 0$ para cualquier evento $A$.
	\item $P(A\cup B) = P(A) + P(B)$ si $A$ y $B$ son disjuntos.
\end{itemize}

\subsection{Probabilidad Frecuentista}

Sea $n_A$ el número de ocurrencias de un evento $A$ en $n$ realizaciones de un experimento aleatorio. La probabilidad frecuentista del evento $A$ se define como el límite
\begin{equation}
	P(A) = \lim _{n\to \infty} \frac{n_A}{n} \label{prob_frecuentista}
\end{equation}

En estadística, a diferencia del análisis matemático, el infinito no tiene sentido; por lo que utilizar el límite es un abuso de notación. Para efectos prácticos se tomará el concepto de "infinito" como una cantidad grande en repeticiones del experimento.

\subsection{Espacios de Probabilidad}

\begin{definicion}
	Un espacio de probabilidad es una terna $(\Omega , \mathscr{F}, P)$, donde $\Omega$ es un conjunto arbitrario, $\mathscr{F}$ es una $\sigma -$álgebra de subconjuntos de $\Omega$, y $P$ es una medida de probabilidad sobre $\mathscr{F}$.
\end{definicion}

Para entender de mejor manera esta definición, es necesario introducir otros conceptos antes, tomaremos a $\mathscr{F}$ como una colección se subconjuntos de $\Omega$.

\begin{definicion}
	{\scshape{Medida de Probabilidad.}} Una función $P$ definida sobre una $\sigma -$álgebra $\mathscr{F}$ y con valores en el interbalo $[0,1]$ es una \textit{medida de probabilidad} si $P(\Omega) = 1$ y es $\sigma -$aditiva, es decir, si cumple que
	$$P\qty( \bigcup _{n=1} ^\infty A_n) = \sum _{n=1} ^\infty P(A_n),$$
	cuando $A_1, A_2, \ldots $ son elementos de $\mathscr{F}$ que cumplen con la condición de ser ajenos dos a dos, esto es, $A_i \cap A_j \neq \emptyset$ para valores de $i$ y $j$ distintos.
\end{definicion}

\begin{definicion}
	{\scshape{Sigma Álgebra.}} Una colección de $\mathscr{F}$ de subonjuntos de $\Omega$ es una $\sigma -$álgebra si
	\begin{enumerate}[a)]
		\item $\emptyset \in \mathscr{F}$.
		\item $\Omega \in \mathscr{F}$.
		\item Si $A \in \mathscr{F}$ entonces $A^c \in \mathscr{F}$.
		\item Si $A_1 ,A_2,\ldots \in \mathscr{F}$ entonces
			$$\bigcup _{n=1} ^\infty A_n \in \mathscr{F}.$$
	\end{enumerate}
\end{definicion}


\begin{ejemplo} \it
	\begin{enumerate}
		\item $\mathscr{F} = \{ \emptyset ,\Omega \}$. \\
			La primera y segunda condición se cumplen. El tercer axioma, dado que tomamos el conjunto universo como $\Omega$, entonces los complementos pertenecen a la $\sigma -$álgebra. El cuarto axioma también se cumple. Entonces $\mathscr{F}$ es una $\sigma -$álgebra.
		\item $\mathscr{F} = \{ \emptyset ,\Omega , A,A^c \}$.
	\end{enumerate}
\end{ejemplo}

Haciendo la analogía, $\Omega$ es el espacio muestral, $\mathscr{F}$ son los eventos.


\begin{teorema}
	Si $A$ y $B\in \mathscr{F}$ y $\mathscr{F}$ es $\sigma -$álgebra entonces $A\cap B \in \mathscr{F}$.
\end{teorema}

\begin{proof}
	Tomando $(A\cap B)^c$, por leyes de DeMorgan, $= A^c \cup B^c$, lo que concluye la prueba.
\end{proof}

¿La intersección infinita de conjuntos está en la $\sigma -$álgebra?
\begin{teorema}
	Si $S$ es una colección de subconjuntos de $\Omega$ y cada uno de los elementos de $S$ pertenecen a una $\sigma -$álgebra $\mathscr{F}$, entonces la intersección de todos los elementos de $S$ pertenece a $\mathscr{F}$.
\end{teorema}

\begin{proof}
	Para el caso contable, se utiliza la idea de la demostración anterior, apliandola por inducción.
\end{proof}


\begin{definicion}
	Sea $S$ una colección de subconjuntos de $\Omega$, entonces la $\sigma -$álgebra generada por $S$ es la menor $\sigma -$álgebra que contiene a $S$.
\end{definicion}



\begin{definicion}
	Sea $\mathscr{C}$ una colección no vacía de subconjuntos de $\Omega$. La $\sigma -$álgebra generada por $\mathscr{C}$, denotada por $\sigma (\mathscr{C})$, es la colección 
		$$\sigma (\mathscr{C}) = \{ \mathscr{F}:\mathscr{F} \quad \sigma -\text{álgebra con } \mathscr{C}\subseteq \mathscr{F} \}$$
\end{definicion}



\begin{definicion}
	{\scshape{Álgebra.}} Una colección $\mathscr{A}$ de subconjuntos de $\Omega$ es una álgebra si cumple las siguientes condiciones:
	\begin{enumerate}
		\item $\Omega \in \mathscr{A}$.
		\item Si $A \in \mathscr{A}$, entonces $A^c \in \mathscr{A}$.
		\item Si $A_1 ,\ldots , A_n \in \mathscr{A}$, entonces $\displaystyle\bigcup _{k=1} ^n A_k \in \mathscr{A}$.
	\end{enumerate}
\end{definicion}
Las $\sigma -$álgebra son subconjuntos de las álgebras.

\begin{definicion}
	{\scshape{Semiálgebra.}} Una colección $\mathscr{S}$ de subconjuntos de $\Omega$ es una semiálgebraa si cumple las siguientes condiciones:
	\begin{enumerate}
		\item $\Omega \in \mathscr{S}$.
		\item Si $A,B\in \mathscr{S}$, entonces $A\cap B \in \mathscr{S}$.
		\item Si $A,A_1 \in \mathscr{S}$ son tales que $A_1 \subseteq A$, entonces existen $A_2 ,\ldots ,A_n \in \mathscr{S}$ tales que los subconjuntos $A_1 ,\ldots ,A_n$ son ajenos dos a dos y se cumple que
			$$A = \bigcup _{k=1} ^n A_k .$$
	\end{enumerate}
\end{definicion}


\begin{definicion}
	{\scshape $\sigma -$Álgebra generada: } Sea $\mathscr{C}$ una colección no vacía de subconjuntos de $\Omega$. La $\sigma -$álgebra generada por $\mathscr{C}$, denotada por $\sigma (\mathscr{C})$
	\dsnote{falta.}
\end{definicion}

\begin{definicion}
	{\scshape $\sigma -$Álgebra de Borel de $\R$: } 
	$$\mathscr{B} (\R) := \sigma \{ (a,b) \subseteq \R:a\leq b \}$$
\end{definicion}

\begin{definicion}
	Sea $A \in \mathscr{B}$. La $\sigma -$álgebra de Borel de $A$, denotada por $\mathscr{B} (A)$ \dsnote{faltaaaaaaaa} o por $A\cap \mathscr{B} (\R)$, se define como sigue
		$$\mathscr{B} (A) = \{ A\cap B \}$$
\end{definicion}

\begin{definicion}
	{\scshape Límite superior e inferior: } Para una sucesión de eventos $A_n : n\in \N$, se define el límite superior y el límite inferior como sigue:
	 	$$\limsup _{n\to \infty} A_n = \bigcap _{n=1} ^\infty \bigcup _{k=1} ^\infty A_k$$
	 	$$\liminf _{n\to \infty} A_n = \bigcup _{n=1} ^\infty \bigcap _{k=1} ^\infty A_k$$
\end{definicion}

\begin{definicion}
	Sea $\{ A_n : n\in \N \}$ una sucesión de eventos. Si existe un evento $A$ tal que
		$$\liminf A_n = \limsup A_n = A$$
	entonces se dice que la sucesión converge al evento $A$.
\end{definicion}






