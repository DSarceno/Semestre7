\section{Distribución de Probabilidad Discreta}
\subsection*{273}
Dada la variable aleatoria $X$ con distribución $\{ 1,\ldots ,n \}$. Calculamos lo siguiente
\begin{itemize}
	\item $\mathbf{E(X)}$ Por definición
		$$E(X) = \sum _{i=0} ^n x_i f(x_i) = \frac{1}{n} \underbrace{\sum _{i=0} ^n x_i}_{\frac{n(n+1)}{2}} = \frac{n+1}{2}.$$
	\item $\mathbf{E(X^2)}$ Por definición
		$$E(X^2) = \sum _{i=0} ^n x_i ^2 f(x_i) = \frac{1}{n} \underbrace{\sum _{i=0} ^n x_i ^2 x_i ^2}_{\frac{n(n+1)(2n+1)}{6}} = \frac{(n+1)(2n+1)}{6} .$$
	\item $\mathbf{V(X)}$ Por definición de varianza
		$$V(X) = E(X^2) - E(X)^2 = \frac{(n+1)(2n+1)}{6} - \frac{n+1}{2},$$
	simplificando con Mathematica (porque ya no estamos para estos trotes xD)
		$$V(X) = \frac{n^2 - 1}{12} .$$
\end{itemize}

\section{Distribución de Probabilidad de Bernoulli}
\subsection*{286}
Teniendo la varianza para la distribución de Bernoulli $V(X) = p(1-p)$, encontramos $p$ para maximizar $V(X)$,
	$$\dv{p} V(X) = 1-2p = 0 \quad \quad \Rightarrow \quad \quad \boxed{p=\frac{1}{2}}.$$
\section{Distribución de Probabilidad Binomial}
\subsection*{305}
Dados los datos $p = 0.9$ y $n = 20$, queremos obtener la probabilidad de no tener el ni el mínimo de éxito, es decir, $x = 18$, por ende queremos la probabilidad acumulada hasta $x = 17$. Utilizando la siguiente función de \textit{R}: \texttt{pbinom(17, size = 20, prob = 0.9)}. Con dicha función se tiene $P(X\leq 17) = \displaystyle\sum _{x=0} ^{17} \binom{20}{x} p^x (1-p)^{20-x} = 0.3230732.$
\section{Distribución de Probabilidad Geométrica}
\section{Distribución de Probabilidad Binomial Negativa}
\subsection*{331}
Dado que necesitamos $n$ lanzamientos y $r = 6$ éxitos; por ende, el número de fracasos $x = n - r$. Sustituyendo en la función de densidad de probabilidad
	$$\boxed{f(n - 6) = \binom{n - 1}{n - 6} \qty(\frac{1}{6})^6 \qty(\frac{5}{6})^{n - 6} .}$$

\section{Distribución de Probabilidad Hipergeométrica}
\subsection*{341}
Dados los datos, se tiene $N = 100$, $K = 90$, $n = 5$ y $x \in [2,n]$. Con esto, dado que queremos la probabilidad de que se realize la compra, calculamos
	$$1 - \sum _{x = 2} ^5 \frac{\binom{90}{x} \binom{10}{5 - x}}{\binom{100}{5}} = 0.9231433.$$
Para esto se utilizó \textit{R}. \\
{\tt
var1 <- 0 \\
for (i in 2:5)\{ \\
var1 <- var1 + dhyper(i,90,10,5) \\
\}
}

\section{Distribución de Probabilidad de Poisson}
\subsection*{343}
Encontrando el valor esperado $E(X)$
\begin{align*}
	E(X) &= \sum _x ^\infty x e^{-\lambda} \frac{\lambda ^x}{x!} \\
	&= \sum _x ^\infty e^{-\lambda} \frac{\lambda ^x}{(x - 1)!}
\end{align*}
\begin{align*}
	&= \lambda e^{-\lambda} \underbrace{\sum _x ^\infty \frac{\lambda ^{x - 1}}{(x - 1)!}}_{e^\lambda} \\
	&= \lambda .
\end{align*}

Ahora, encontrando la varianza $V(X) = E(X^2) - E(X)^2$, encontramos la esperanza de $X^2$
	\begin{align*}
		E(X^2) &= \sum _x x^2 e^{-\lambda} \frac{\lambda ^x}{x!} \\
		&= \sum _x \lambda xe^{-\lambda} \frac{\lambda ^{x - 1}}{(x - 1)!} \\
		&= \lambda e^{-\lambda} \sum _j (j + 1) \frac{\lambda ^j}{j!} \\
		&= \lambda e^{-\lambda} \qty[\underbrace{\sum _j j\frac{\lambda ^j}{j!}}_{\lambda e^\lambda} + \underbrace{\sum _j \frac{\lambda ^j}{j!}}_{e^\lambda}] \\
		&= \lambda ^2 + \lambda ,
	\end{align*}

entonces
\begin{align*}
	V(X) &= \lambda ^2 + \lambda - \lambda ^2 \\
	&= \lambda .
\end{align*}



\section{Distribución de Probabilidad Uniforme Continua}
\subsection*{357}

\begin{enumerate}[a)]
	\item Entontrando $E(X)$, se tiene que $f(x) = \frac{1}{b-a}$, entonces
		$$E(X) = \int _a ^b \frac{x}{b-a} \dd{x} = \frac{1}{b-a} \frac{b^2 - a^2}{2} = \frac{a+b}{2}.$$
	\item Encontrando $E(X^2)$, entonces, realizando la integral
		$$\int _a ^b \frac{x^2}{b-a} \dd{x} = \frac{1}{b-a} \frac{b^3 - a^3}{3} = \frac{b^2 + ab + a^2}{3}.$$
	\item Encontrando la varianza,
		$$Var(X) = E(X^2) - E(X)^2 = \frac{a^2 + ab + b^2}{3} - \frac{(a + b)^2}{4} = \frac{a^2 + b^2}{12} - \frac{ab}{6} = \frac{(a - b)^2}{12}.$$
\end{enumerate}


\section{Distribución de Probabilidad Exponencial}
\subsection*{375}
Encontramos el enésimo momento, $E(X^n)$, realizando la integral
	$$E(X^n) = \lambda \int _0 ^\infty x^n e^{-\lambda x} \dd{x},$$
realizando el cambio de variable $\lambda x = t$, se tiene
	$$\int _0 ^\infty \qty(\frac{t}{\lambda})^n e^{-t} \dd{t} = \frac{1}{\lambda ^n} \int _0 ^\infty t^n e^{-t} \dd{t},$$
la cual es exactamente a la función Gamma valuada en $n+1$, $\Gamma (n+1) = n!$, por lo tanto
	$$E(X^n) = \frac{\Gamma (n+1)}{\lambda ^n} = \frac{n!}{\lambda ^n}.$$















%%%