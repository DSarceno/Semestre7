\section{Experimento Michelson$-$Morley}

\subsection{Contexto Histórico}

A finales del siglo $XIX$, se tenía la idea de que, así como las ondas sonoras (Mecánicas en general) requieren de un medio para propagarse (como el agua o aire), la velocidad de la luz también lo requiere. Además, aún se tenía la idea de un marco de referencia "privilegiado", o universal, sobre el cual se pueden realizar todas las medidas y sobre el cual se mueven el resto de objetos en el espacio. A este medio se le llamó "éter".

\subsection{Explicación del Experimento}

En ese tiempo se creía que, la dirección del éter variaría al medirse desde la tierra, por lo que el experimento se realizaría en varios momentos del año. Dado esto, se suponía una dirección para el éter y se utilizó un, ahora conocido como, interferometro de Michelson, el cual es una lente semiplateada ("half$-$silvered mirror"), la cual divide la luz monocromatica en dos haces de luz. Este aparato se coloca de modo que uno de los haces de luz resultante sea perpendiular a la corriente del éter y el otro paralelo a ella. Estos haces serán reflejados en dos espejos, uno para cada uno, y regresarán al mismo punto en el interferometro de Michelson, pero en fase destructiva, es decir, un haz estaría retrasado respecto al otro. Aunque los instrumentos utilizados en el experimento de Michelson y Morley eran lo sufiecientemente sensibles, esta fase entre los haces, no se encontró.

\subsection{Consecuencias}

El resultado negativo del experimento tuvo dos importantes consecuencias:
	\begin{description}
		\item[Primero: ] El éter no existe, es decir, todo movimiento (suceso) ocurre  respecto a un marco de referencia en específico, no respecto a uno privilegiado o universial.
		\item[Segundo: ] La velocidad de la luz es la misma para todos los observadores, a diferencia de otros tipos de onda que requieren un medio de propagación.
	\end{description}
No pasaría mucho tiempo, para que fuera demostrado de manera teórica. En $1905$ publicara su artículo referido a la relatividad especial, con la salvedad de que sus hallazgos surgen de un problema de simetría en electromagnetismo.