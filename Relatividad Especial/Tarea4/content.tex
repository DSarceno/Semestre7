\section{Transformación Propia de Lorentz}

Tomando la transformación propia de Lorentz $L_{+\uparrow} (\beta)$ aplicada al $4-$vector
	\begin{equation*}
		L_{+\uparrow} (\beta) \mqty(t \\ \vec{r}) = \mqty( \gamma & -\gamma \beta ^t \\ -\gamma \beta & I + \frac{\gamma - 1}{\beta ^2} \beta \beta ^t ) \mqty(t \\ \vec{r}) = \mqty(t\gamma - \gamma \vec{\beta} ^t \vec{r} \\ -\gamma t \vec{\beta} + \vec{r} + \frac{\gamma - 1}{\vec{\beta} ^t} \vec{\beta} \qty(\vec{\beta} ^t \vec{r})).
	\end{equation*}
Dado que $\vec{\beta}$ es un vector columna, el producto $\vec{\beta} ^t \vec{r} = \vec{\beta} \cdot \vec{r}$, con esto, se obtienen las dos relaciones buscadas
	\begin{equation}
		\left\{\begin{array}{c}
			t' = \gamma \qty(t -  \vec{\beta} \cdot \vec{r}) \\
			\vec{r} ' = -\gamma t \vec{\beta} + \vec{r} + \frac{\gamma - 1}{\vec{\beta} ^t} \vec{\beta} \qty(\vec{\beta} \cdot \vec{r})
		\end{array}\right. . \label{TransPropV}
	\end{equation}
	
Para la segunda parte, sabiendo que $A^i = \mqty(t^+ \\ \vec{0})$, utilizando el resultado temporal de \eqref{TransPropV} y dado que $\vec{r} = \vec{0}$, entonces
	\begin{equation}
		t^+ = \gamma t,
	\end{equation}
lo que implica que $t^+$ y $t$ tienen el mismo signo.