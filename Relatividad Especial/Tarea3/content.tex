\section{Tema 1: Matrices Isométricas}

\textit{Demostrar que $A^T = A^{-1}$ se cumple para la siguiente matriz}
	$$A = \mqty(a_1 & a_2 \\ a_2 & a_4).$$

Esto es claro invocando el teorema espectral para espacios con producto interno sobre los reales. El teorema espectral enuncia que para un operador lineal $T$, los siguientes argumentos son equivalentes

	\begin{enumerate}
		\item $T$ es hermítico. (En un espacio sobre los reales es equivalente a ser simétrico)
		\item El espacio tiene una base ortonormal formada por vectores propios de $T$.
		\item $T$ tiene una matriz diagonal respecto a alguna base ortonormal del espacio.
	\end{enumerate}
	
Sin embargo, es necesaria una condicion extra, el determinante de la matriz debe ser $\pm 1$. Dado todo esto, es claro que la matriz es una isometría, y todo lo que ello conyeva, es decir, $A^\dagger = A^{-1}$.