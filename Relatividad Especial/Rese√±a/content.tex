Reseña y solución de los ejercicios de la sección $1.7$ del libro de Tsamparlis.

\section{Tema 1}

La idea de esta sección es encontrar la transfomración del Lorentz de una manera algebraica. Esto inicia dada la ecuación
	\begin{equation}
		\eta = L^t \eta L, \label{EcuacionMatricial}
	\end{equation}
donde $\eta = \text{diag} (-1,1,1,1)$. Se supone una matriz arbitraria de bloques en el espacio cartesiano lorentziano
	\begin{equation}
		L = \mqty(D & A \\ B & A) \label{MatrizBloques}
	\end{equation}
con $D$ una matriz $1\cp 1$, $B$ un vector columna, $C$ un vector fila y $A$ una matriz $3\cp 3$. Sustituyendo esta trasformación en \eqref{EcuacionMatricial} se llega a un sistema de ecuaciónes matricial; el cual, para resolverse, es necesario dividirse en casos.
	\begin{equation}
		\mqty{A^t A - C^t C = I_3 \\ B^t A - DC = 0 \\ B^t B - D^2 = -1} \label{SistemaMatricial}
	\end{equation}
\begin{description}
	\item[Caso 1: ] $C = 0$ y $A \neq 0$, lo que nos deja que $A^t A = I_3$, $B = 0$ y $D = \pm 1$. Con esto se construyen las transformaciones
		\begin{equation}
			R_+ (E) = \mqty(1 & 0 \\ 0 & E) \quad \quad \quad R_- (E) = \mqty(-1 & 0 \\ 0 & E), \label{EOT}
		\end{equation}
		las cuales son trasformaciones euclideanas ortogonales, las cuales cumplen con las siguientes relaciones:
			\begin{equation}
				\left\{\begin{array}{c}
					\text{det} R_\pm (E) = \pm 1 \\
					R_+ ^t (E) R_+ (E) = R_- (E) ^t R_- (E) = I_4 \\
					R_+ (E) ^t R_- (E) = \eta
				\end{array}\right. \label{EOTRelaciones}
			\end{equation}
	\item[Caso 2: ] Para este caso se define $A = \text{diag} (K,1,1), \abs{K} \geq 1$, $B^t = (B_1 ,B_2 ,B_3)$, $C = (C_1 ,C_2 ,C_3)$; dado esto, y tomando \eqref{SistemaMatricial} implica que:
	\begin{equation}
		\left\{ \mqty{C_1 = \pm \sqrt{K^2 - 1}, \, C_2 = C_3 = 0 \\ B_1 = \pm \frac{D}{K} \sqrt{K^2 -1}, \, B_2 = B_3 = 0 \\ D = \pm \abs{K}} \right. . \label{ConclusionesCaso2}
	\end{equation}
	Esto define las siguientes transformaciones de Lorentz
	\begin{align*}
		L_1 = \mqty( K & C_1 & 0 & 0 \\ C_1 & K & 0 & 0 \\ 0 & 0 & 1 & 0 \\ 0 & 0 & 0 & 1 ), &\quad L_2 = \mqty( -K & -C_1 & 0 & 0 \\ C_1 & K & 0 & 0 \\ 0 & 0 & 1 & 0 \\ 0 & 0 & 0 & 1 ), \\ 
		L_3 = \mqty( K & C_1 & 0 & 0 \\ -C_1 & K & 0 & 0 \\ 0 & 0 & 1 & 0 \\ 0 & 0 & 0 & 1 ), &\quad L_4 = \mqty( -K & -C_1 & 0 & 0 \\ -C_1 & K & 0 & 0 \\ 0 & 0 & 1 & 0 \\ 0 & 0 & 0 & 1 ).
	\end{align*}
	Las soluciones especiales son llamadas \textit{boosts}.
	\item[Solución General: ] Para esto se definió $\beta$ como $B = -D\beta$, dado esto y utilizando el sistema \eqref{SistemaMatricial}, dado esto $\beta$ debe cumplir que $0 < \beta ^2 < 1$. Realizando un poco de algebra matricial se concluye que 
	\begin{align*}
		\left\{\begin{array}{c}
			A = \pm \qty(I + \frac{D - 1}{\beta ^2} \beta \beta ^t) E, \\
			B = \mp D\beta , \\
			C = \mp D\beta ^t ,
		\end{array}\right.
	\end{align*}
	con $E$ una matriz euclideana ortogonal, entonces se concluye la transformación general de Lorentz
		\begin{equation}
			L(\beta ,E) = L(\beta) R(E), \label{TransGeneral}
		\end{equation}
	donde la matriz $R(E)$ es la solución del caso $1$. 
		\begin{equation}
			L(\beta) = \mqty( \pm \gamma & \mp \gamma \beta ^t \\ \mp \gamma \beta & \pm \qty(\delta ^\mu _\nu + \frac{\text{det} L(\gamma - 1)}{\beta ^2} \beta \beta ^t) ) .
		\end{equation}
		Con $D = \pm \gamma$ y $\gamma = \frac{1}{\sqrt{1-\beta ^2}}$. Si se toman todos los casos de las matrices euclideanas ortogonales se tendrían $8$ casos en total. Y, nos interesan cuatro casos
		\begin{enumerate}[a)]
			\item Trans. propia de Lorentz ($D = \gamma$):
				\begin{equation}
					L_{+\uparrow} (\beta) = \mqty( \gamma & -\gamma \beta ^t \\ -\gamma \beta & I + \frac{\gamma - 1}{\beta ^2} \beta \beta ^t )
				\end{equation}
			\item Trans. de Lorentz con inversión espacial ($D = \gamma$):
				\begin{equation}
					L_{-\downarrow} (\beta) = \mqty( \gamma & \gamma \beta ^t \\ -\gamma \beta & -I - \frac{\gamma - 1}{\beta ^2} \beta \beta ^t )
				\end{equation}
			\item Trans. de Lorentz con inversión temporal ($D = -\gamma$):
				\begin{equation}
					L_{-\uparrow} (\beta) = \mqty( -\gamma & \gamma \beta ^t \\ \gamma \beta & I - \frac{\gamma - 1}{\beta ^2} \beta \beta ^t )
				\end{equation}
			\item Trans. de Lorentz con inversión espacio-temporal ($D = -\gamma$):
				\begin{equation}
					L_{+\downarrow} (\beta) = \mqty( -\gamma & -\gamma \beta ^t \\ \gamma \beta & -I - \frac{\gamma - 1}{\beta ^2} \beta \beta ^t )
				\end{equation}
		\end{enumerate}
\end{description}

Las cuatro transformaciones mostradas nos son útiles en física, pero nos ineteresarán más las transformaciones propias de Lorentz ya que forman un grupo. Un subgrupo de estas transformaciones es el llamado \textbf{boosts} formado por las matrices $L_{+\uparrow ,i} (\beta)$ donde $i$ es el eje. Bajo esta idea, se llega a las transformaciones propias de lorentz, construídas por esta ecuación matricial
	\begin{equation}
		\mqty(t^\prime \\ \vb{r}^\prime) = L(\beta) \mqty(t \\ \vb{r}) \label{properTrans}
	\end{equation}

\section{Ejercicios}

\subsection{1.7.1}
Demostramos las cuatro identidades mostradas:
\begin{enumerate}
	\item $\gamma ^2 = \gamma ^2 \beta ^2 + 1$, tomamos el lado derecho de la ecuación y desarrollamos con el denominador
	$$\gamma ^2 \beta ^2 + 1 = \frac{\beta ^2}{1 - \beta ^2} + 1 = \frac{\beta ^2 + 1 - \beta ^2}{1 - \beta ^2} = \frac{1}{1 + \beta ^2} = \gamma ^2 .$$
	$\hfill \square$
	\item $\gamma \qty(\displaystyle\frac{\beta _1 \pm \beta _2}{1 \pm \beta _1 \beta _2}) = \gamma (\beta _1) \gamma (\beta _2) \qty(1 \pm \beta _1 \beta _2)$, tomando el lado izquierdo de la ecuación, para simplicidad y facilidad en la escritura, utilizamos directamente el $1-$ el término valuado:	
	$$
			1-\qty(\frac{\beta _1 \pm \beta _2}{1\pm \beta _1 \beta _2}) = \frac{1\pm 2\beta _1 \beta _2 + (\beta _1 \beta _2)^2 - \qty(\beta _1 ^2 \pm 2\beta _1 \beta _2 + \beta _2 ^2)}{\qty(1 \pm \beta _1 \beta _2)^2} = \frac{(1 - \beta _1 ^2)(1 - \beta _2 ^2)}{\qty(1 \pm \beta _1 \beta _2)^2},
	$$
	sustituyendo esto en la definición de $\gamma$ se tiene 
	$$
		\frac{1 \pm \beta _1 \beta _2}{\sqrt{1 - \beta _1 ^2}} \frac{1}{\sqrt{1 - \beta _2 ^2}} = \gamma (\beta _1) \gamma (\beta _2) \qty(1 \pm \beta _1 \beta _2).
	$$
	$\hfill \square$
	\item $\dd{\gamma} = \gamma ^3 \beta \dd{\beta}$, utilizando la regla de la cadena
	$$
		\dd{\gamma} = \frac{-1}{2} \gamma ^3 (-2\beta \dd{\beta}) = \gamma ^3 \beta \dd{\beta} .
	$$
	$\hfill \square$
	\item $\dd{(\gamma \beta)} = \gamma ^3 \dd{\beta}$, nuevamente, por la regla de la cadena
	$$
		\dd{(\gamma \beta)}	= \qty(\beta ^2 \gamma ^3 + \gamma) \dd{\beta} = \gamma ^3 \dd{\beta}.
	$$
	$\hfill \square$
	\item Este inciso es claro, es la serie de Taylor alrededor de $\beta = 0$, de ahí
	$$
		\gamma = 1 + \frac{1}{2} \beta ^2 + \frac{3}{8} \beta ^4 + \cdots	
	$$
	$\hfill \square$
\end{enumerate}


\subsection{1.7.2}
Tomando la transformación dada y valuandola en $-\beta$, se tiene
	$$
		L^{i'} _j (-\beta) L^{i'} _j (\beta) = \mqty(\dmat[0]{\gamma ^2 - \beta ^2 \gamma ^2, \gamma ^2 - \beta ^2 \gamma ^2, 1, 1}),
	$$
	donde, por el ejercicio anterior, $\gamma ^2 - \beta ^2 \beta ^2 = 1$, lo que nos da la identidad, entonces $L ^{i'} _j (-\beta) = \qty(L ^{i'} _j (\beta))^{-1}$. Para demostrar que es una transformación de Lorentz, utilizamos $\eta = L^t \eta L$, realizando la multiplicación con mathematica, se llega a la matriz anterior, con la salvedad de que el primer término es $-1$, lo cual es $\eta$, por lo que si es una transformación de Lorentz. \\
Para el último inciso, dado que $\sinh{\phi} = \beta \gamma$ y $\cosh{\phi} = \gamma$, entonces es claro, por definición, que 
	$$\tanh{\phi} = \frac{\sinh{\phi}}{\cosh{\phi}} = \beta .$$
También, por definición de funciones trigonométricas hiperbólicas se suman dichas funciónes, con lo que se tiene
	$$\sinh{\phi} + \cosh{\phi} = \beta \gamma + \gamma = \frac{e^\phi - e^{-\phi}}{2} + \frac{e^\phi + e^{-\phi}}{2} = e^\phi ,$$
	Reduciendo la expresión $\gamma *(1 + \beta) = \sqrt{\frac{1 + \beta}{(1 + \beta)(1 - \beta)}} = \sqrt{\frac{1 + \beta}{1 - \beta}} \hfill \square$
	
\subsection{1.7.3}
Tomando las matrices dadas en el ejemplo anterior, se tiene
\begin{align*}
	[L] &= \left(
	\begin{array}{cccc}
		 \cosh (\phi) & -\sinh (\phi) & 0 & 0 \\
		 -\sinh (\phi) & \cosh (\phi) & 0 & 0 \\
		 0 & 0 & 1 & 0 \\
		 0 & 0 & 0 & 1 \\
	\end{array}
	\right) \\
	[V]_K &= \mqty(-\sinh (\phi) \\ \cosh (\phi)\\ 0\\ 1) \\
	[T]_K &= \left(
		\begin{array}{cccc}
			 \sinh ^2(\phi) & \sinh (\phi) (-\cosh (\phi)) & 0 & 0 \\
			 \sinh (\phi) (-\cosh (\phi)) & \cosh ^2(\phi) & 0 & 0 \\
			 0 & 0 & 0 & 0 \\
			 0 & 0 & 0 & 1 \\
		\end{array}
		\right)  \\
	[L^{-1}] &= \left(
		\begin{array}{cccc}
			 \cosh (\phi) & \sinh (\phi) & 0 & 0 \\
			 \sinh (\phi) & \cosh (\phi) & 0 & 0 \\
			 0 & 0 & 1 & 0 \\
			 0 & 0 & 0 & 1 \\
		\end{array}
		\right) .
\end{align*}
	
Con estas matrices se realizan las operaciones propuestas, se utilizó mathematica\footnote{El notebook de mathematica creado lo puede encontrar en \href{https://github.com/DSarceno/Semestre7/tree/main/Relatividad\%20Especial/Rese\%C3\%B1a}{GitHub}} para facilitar la operatoria
	\begin{align*}
		[V]_{K'} &= [L][V]_K = \mqty(-2 \sinh (\phi) \cosh (\phi) \\ \sinh ^2(\phi)+\cosh ^2(\phi) \\ 0 \\ 1) \\
		[T]_{K'} &= [L^{-1}]^t [T]_K [L^{-1}] = \left(
			\begin{array}{cccc}
				 0 & 2\sinh (\phi) \cosh (\phi) & 0 & 0 \\
				 0 & \cosh ^2 (\phi) \sinh ^2 (\phi) & 0 & 0 \\
				 0 & 0 & 0 & 0 \\
				 0 & 0 & 0 & 1 \\
			\end{array}
			\right) \\
		[a]_K &= \qty([T]^t [V]_K)^t = \mqty(-\sinh ^3(\phi)-\sinh (\phi) \cosh ^2(\phi) \\ \cosh ^3(\phi)+\sinh ^2(\phi) \cosh (\phi) \\ 0 \\ 1) \\
		a_i V^i &= [a]_K [V]_K = \frac{1}{2} \qty(3 + \cosh{4\phi})
	\end{align*}



































%%%%