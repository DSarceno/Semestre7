\section{Problema 1}
Teniendo el sistema $A^{(0)}$ compuesto por $A$ y $A'$ y sabiendo que $A$ es muy pequeño en comparaión con $A'$, por conservación de la energía	
	$$E_r + E' = E^{(0)}.$$
Entonces, cuanto $A$ tiene una energía $E_r$ $A'$ tiene energía $E' = E^{(0)} - E_r$. Si $A$ esta en un estado definido $r$ el número de estados accesibles para $A'$ en el sistema total $A^{(0)}$ es
	$$\Omega ' (E^{(0)} - E_r);$$
además, se postula que en una situación de equilibrio es igualmente probable que el sistema se encuentre en cualquiera de sus estados accesibles $P_r = C$, entonces
	$$P_r = C' \Omega ' (E^{(0)} - E_r).$$
Como $A$ es mucho menor a $A'$ podemos aproximar el número de estados accesibles, utilizando logaritmos para poder utilizar las series de Taylor. Como $E_r \ll E^{(0)}$, realizamos la siguiente expansión
	$$\ln{\Omega ' (E^{(0)} - E_r)} \approx \ln{\Omega '(E^{(0)})} - E_r \pdv{\ln{\Omega ' (E^{(0)})}}{E'} + \cdots ,$$
y definimos a $\beta$ como 
	$$\beta = \pdv{\ln{\Omega ' (E^{(0)})}}{E'}.$$
Dada la expansión realizada se tiene
	$$\Omega '(E^{(0)} - E_r) = \underbrace{\Omega '(E^{(0)})}_{C'} e^{-\beta E_r},$$
entonces
	$$P_r = C' e^{-\beta E_r}.$$
Sabiendo que $\sum _r P_r = 1$, encontramos la constante de normalización $\partition = \sum _r e^{-\beta E_r}$, entonces es claro que
	$$\boxed{P_r = \frac{e^{-\beta E_r}}{\displaystyle\sum _i e^{-\beta E_i}}}$$
	
\section{Problema 2}
Sabiendo que el valor esperado de la energía es $\langle E_i \rangle = U$, entonces tomando la función de partición y derivando la respecto a $\beta$
	$$ \dv{\partition}{\beta} = \sum _{i = 1} ^k -E_i e^{-\beta E_i}, $$
	multiplicando y dividiendo entre la función de partición se tiene
	$$\frac{1}{\partition} \dv{\partition}{\beta} = -\sum _{i = 1} ^k p_i E_i = -U .$$
	Por regla de la cadena 
	$$\dv{\ln{\partition}}{\beta} = \dv{\ln{\partition}}{\partition} \dv{\partition}{\beta} = \frac{1}{\partition} \dv{\partition}{\beta}.$$
	$$\dv{\ln{\partition}}{\beta} = -U.$$
	Teniendo esto, y sabiendo que $\beta = \flatfrac{1}{k_B T}$, encontramos el diferencial de esta expresión $\dd{\beta} = \flatfrac{-\dd{T}}{k_B T^2}$, sustituyendo el diferencial en la expresión anterior
		$$\boxed{k_B T^2 \dv{\ln{\partition}}{T} = U}$$
	
\section{Problema 3}
Tomando la entropía 
	$$S = -k_B \sum _{i = 1} ^k p_i \ln{p_i},$$
sustituyendo la probabilidad definida como
	$$\ln{p_i} = -\beta E_i - \ln{\partition}.$$
Sustituyendo en la ecuación de entropía se tiene
	$$S = k_B \beta \underbrace{\sum_{i = 1} ^k p_i E_i}_{U} + k_B \ln{\partition} \underbrace{\sum _{i = 1} ^k p_i}_{1},$$
	entonces
	\begin{equation}
		S = k_B \qty(\beta U + \ln{\partition}) \label{Entropy}
	\end{equation}
	Teniendo \eqref{Entropy} sustituímos la definición de $\beta$, entonces
	$$\boxed{S = \frac{U}{T} + k_B \ln{\partition}}$$
	
\section{Problema 4}
Tomando la relación obtenida en el ejercicio anterior, despejamos la función $\partition$,
	$$TS = U + \frac{1}{\beta} \ln{\partition} ,$$
	$$-\beta \qty(U - TS) = \ln{\partition},$$
	entonces
	$$\boxed{\partition = e^{-\beta F}}$$
	con $F = U - TS$.

\section{Problema 5}
Encontrando el diferencial de la función de Helmholtz, sabiendo que $\dd{U} = T\dd{S} - p\dd{V}$, entonces 
	$$\dd{F} = T\dd{S} - p\dd{V} - T\dd{S} - S\dd{T} = -S\dd{T} - p\dd{V}.$$
Lo que implica directamente que
	$$\boxed{S = -\qty(\pdv{F}{T})_V}$$
























%%%%%
