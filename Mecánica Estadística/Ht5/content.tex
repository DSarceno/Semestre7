\section{Teorema de Equipartición}
Si la energía de un sistema clasico es la suma de $n$ modos cuadráticos, y ese sistema está en contacto con una fuente de calor a una temperatura $T$, la energía media del sistema esta dada por $\frac{n}{2} k_B T$. (\textit{Blundell and Blundell.})

\section{Problema 1}
Dada la energía
	$$ E = \sum _{k = 1} ^3 \frac{1}{2} mv_k ^2, $$
la energía media utilizando el teorema de equipartición se tiene
	$$ \boxed{\langle E \rangle = \frac{3}{2} k_B T.} $$




\section{Problema 2}
Dada la energía para el modelo de la molécula y diatómica
	$$ E = \sum _{k = 1} ^3 \frac{1}{2} mv_k ^2 + \sum _{k = 1} ^2 \frac{L_k ^2}{2I_k} + \frac{1}{2} \mu (\dot{r} _1 - \dot{r}_2)^2 + \frac{1}{2} k(r_1 - r_2)^2, $$
dado que se tienen solo modos cuadrados, entonces la energía media es
	$$ \boxed{\langle E \rangle = \frac{7}{2} k_B T.} $$