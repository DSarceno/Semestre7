\section{Problema 1}
Dada la varianza $\langle (\Delta E)^2 \rangle = \langle (E_i - \varepsilon)^2 \rangle = \langle E_i ^2 \rangle - \varepsilon ^2$, entonces, encontramos el segundo momento
\begin{equation}
	 \dv[2]{\partition}{\beta} = \sum E_i ^2 e^{-\beta E_i} = \underbrace{\qty(\sum p_i E_i ^2)}_{\langle E_i ^2 \rangle} \partition .  \label{E2}
\end{equation}
Ahora, tomando la segunda derivada de $\ln{\partition}$, se tiene
	$$ \dv[2]{\ln{\partition}}{\beta} = \dv{\beta} \qty(\frac{1}{\partition} \dv{\partition}{\beta}) = -\frac{1}{\partition ^2} \qty(\dv{\partition}{\beta})^2 + \frac{1}{\partition} \dv[2]{\partition}{\beta}, $$
despejando y reemplazando en la fórmula de la varianza
	$$ \langle (\Delta E)^2 \rangle = \dv[2]{\ln{\partition}}{\beta} + \frac{1}{\partition ^2} \qty(\dv{\partition}{\beta})^2 - \qty(-\frac{1}{\partition} \dv{\partition}{\beta})^2 $$
	$$ \boxed{\langle (\Delta E)^2 \rangle = \dv[2]{\ln{\partition}}{\beta} .} $$

\section{Problema 2}
Dada la definición de tercer momento, se realiza la expación
	\begin{equation}
		\langle (\Delta E)^3 \rangle = \sum \underbrace{(E_i - \varepsilon)^3}_{E_i ^3 -3E_i ^2 \varepsilon + 3E_i \varepsilon ^2 - \varepsilon ^3} p_i = \langle E_i ^3 \rangle - 3\varepsilon \langle E_i ^2 \rangle + 3\varepsilon ^3 - \varepsilon ^3 = \langle E_i ^3 \rangle - 3\varepsilon \langle E_i ^2 \rangle + 2\varepsilon ^3. \label{3ermomento}
	\end{equation}
Ahora, siguiendo la idea del problema anterior
	$$ -\frac{1}{\partition} \dv[3]{\partition}{\beta} = \langle E_i ^3 \rangle, $$
entontrando la tercera derivada de $\ln{\partition}$,
	\begin{equation}
		\dv[3]{\ln{\partition}}{\beta} = \frac{2}{\partition ^3} \qty(\dv{\partition}{\beta})^3 - \frac{3}{\partition ^2} \qty(\dv{\partition}{\beta}) \dv[2]{\partition}{\beta} + \frac{1}{\partition} \dv[3]{\partition}{\beta}. \label{dv3ln}
	\end{equation}

Sustituyendo \eqref{E2}, \eqref{dv3ln} y $\varepsilon$ en \eqref{3ermomento}
	$$ \langle (\Delta E)^3 \rangle = - \dv[3]{\ln{\partition}}{\beta} + \frac{2}{\partition ^3} \qty(\dv{\partition}{\beta})^3 - \frac{3}{\partition ^2} \qty(\dv{\partition}{\beta}) \dv[2]{\partition}{\beta} - 3\qty(-\frac{1}{\partition} \dv{\partition}{\beta}) \qty(\frac{1}{\partition} \dv[2]{\partition}{\beta}) + 2\qty(-\frac{1}{\partition} \dv{\partition}{\beta})^3, $$
	$$ \boxed{\langle (\Delta E)^3 \rangle = - \dv[3]{\ln{\partition}}{\beta}.} $$

\section{Problema 3}
Dada la definición de calor específico
	$$c_V = \pdv{\varepsilon}{T},$$
entonces, tomando la energía promedio	
	$$ \varepsilon = - \pdv{\ln{\partition}}{\beta} = k_B T^2 \pdv{\ln{\partition}}{T}. $$
	
Reemplazando en la definición de calor específico
	$$ c_V = \pdv{T} \qty(k_B T^2 \pdv{\ln{\partition}}{T}), $$
	$$ \boxed{c_V = 2k_B T\pdv{\ln{\partition}}{T} + k_B T \pdv[2]{\ln{\partition}}{T}.} $$
	
	
\section{Problema 4}
Sabiendo que la presión la podemos escribir en términos de la energía libre de Helmholtz
	$$ P = -\qty(\pdv{A}{V})_T, $$
tomando a $A$ como 
	$$ A = -\frac{\ln{Z}}{\beta}, \qquad Z = \partition ^n. $$
Reemplazando $A$ en la presión, se tiene
	$$ P = \qty(\ln{\partition} \cancelto{0}{\pdv{T}{V}} \pdv{T} \frac{1}{\beta} + \frac{1}{\beta} \pdv{\ln{Z}}{V})_T, $$
entonces
	$$ \boxed{ P = k_B T \qty(\pdv{\ln{Z}}{V})_T = nk_B T \qty(\pdv{\ln{\partition}}{V})_T .} $$
	

















%%%%%
